\documentclass{article}
\usepackage[margin=1in]{geometry}

\usepackage{fontspec}
\setmainfont{Noto Sans}

\usepackage{xeCJK}
\setCJKmainfont{Noto Sans CJK TC}

\newfontfamily\arabicfont{Noto Kufi Arabic}
\newfontfamily\cherokeefont{Noto Sans Cherokee}
\newfontfamily\devanagarifont{Noto Sans Devanagari}
\newfontfamily\kannadafont{Noto Sans Kannada}
\newfontfamily\khmerfont{Noto Sans Khmer}
\newfontfamily\myanmarfont{Noto Sans Myanmar}
\newfontfamily\tamilfont{Noto Sans Tamil}
\newfontfamily\thaifont{Noto Sans Thai}

\usepackage{microtype}
\usepackage{tabularx}
\usepackage{multirow}
\usepackage{hyperref}

\newcommand{\dfn}[4]{\noindent\textbf{#1}: #2\quad(#3)\hfill\emph{#4}\\}
\newcommand{\ex}[2]{\noindent#1\bigskip\par#2\bigskip}

\title{Chowita}
\author{KeyboardFire $\langle$\href{mailto:andy@keyboardfire.com}{andy@keyboardfire.com}$\rangle$}

\begin{document}
\maketitle
\tableofcontents
\newpage

\section{Introduction}

Chowita (\textbf{tcowitha}, pronounced /t͜ʃowitʰa/) is\ldots TODO

\section{Phonology}

\subsection{Phonemic inventory}

\begin{tabular}{lllllll}
    && bilabial & alveolar & postalveolar & velar & glottal \\
    nasal && (m) & n & (n̠) & (ŋ) \\
    \multirow{3}{*}{stop} & tenuis & p & t & & k & ʔ \\
    & aspirated & pʰ & tʰ & & kʰ \\
    & voiced & b & d & & g \\
    \multirow{2}{*}{fricative} & voiceless & f & s & ʃ & x & h \\
    & voiced & v & z & ʒ & ɣ \\
\end{tabular}\bigskip

vowels: a e i o u ə

approximants: j w

\subsection{Phonotactics}

The syllable structure is \textbf{(C)(C)(G)V(G)(N)}, where \textbf{C}
represents a consonant, \textbf{G} represents a glide (/j/ or /w/), \textbf{V}
represents a vowel, and \textbf{N} represents a nasal.

Consonant clusters are always [stop][fricative]; both must be of the same
voicing, and aspirated stops are not allowed in clusters.

Trailing nasals assimilate to the following place of articulation (or [n] if it
is glottal). For example, \textbf{gankce} ``swim'' is pronounced /gaŋ.k͜ʃe/ with
a velar nasal.

The second (G) may not be /w/.

\subsection{Allophony}

Velars are in free variation with uvulars, most notably among the fricatives.
The voiced velar fricative may also be realized as an approximant or uvular
trill.

After a consonant, the semivowels w and j are typically realized as rounding
(Xʷ) and palatalization (Xʲ) respectively of the previous consonant.
Furthermore, ʃʲ and ʒʲ are frequently realized as ɕ and ʑ.

Aspirated stops may be freely replaced with ejectives, and voiced stops with
implosives.

\section{Orthography}

\begin{tabular}{lllllll}
    && bilabial & alveolar & postalveolar & velar & glottal \\
    nasal && & n \\
    \multirow{3}{*}{stop} & tenuis & p & t & & k & ' \\
    & aspirated & ph & th & & kh \\
    & voiced & b & d & & g \\
    \multirow{2}{*}{fricative} & voiceless & f & s & c & x & h \\
    & voiced & v & z & j & gh \\
\end{tabular}\bigskip

vowels: a e i o u y

approximants: y w

\section{Morphology}

All root words are one syllable of the form \textbf{(C)C(G)V(G)(N)}.
Additionally, the consonants will never be glottals (h or ').

Words are frequently modified by infixes. There are two types of infixes:

\begin{itemize}

    \item \emph{Glottal infixes} insert a vowel followed by a glottal stop
        directly before the vowel in a root. The syntax that this document will
        use to represent glottal infixes is, for example, a$_g$, which
        represents an infix of \textbf{a} followed by a glottal stop.

        As an example of the usage of glottal infixes, the infix i$_g$ applied
        to the word \textbf{kway} results in the modified word \textbf{kwi'ay}.

    \item \emph{Approximant infixes} insert a vowel followed by an approxmant
        which varies depending on the vowel that follows it. Before an
        unrounded vowel, the approximant \textbf{y} /j/ is used. Before a
        rounded vowel, \textbf{w} /w/ is used. The syntax for this is a$_a$.

        For example, the infix e$_a$ applied to the word \textbf{pci} is
        \textbf{pcewi}, but applied to the word \textbf{bzu}, it becomes
        \textbf{bzeyu}. Note the usage of \textbf{w} before \textbf{i} and
        \textbf{y} before \textbf{u}.

\end{itemize}

\section{Syntax}

\ex{%
Chowita is an SVO language. All phrases have the following structure:
}{
[subject-a$_a$] [verb-a$_a$] [direct object]
}

\ex{%
Phrases may be used as nested subjects for other phrases:
}{
[phrase-a$_a$] [verb-a$_a$] [direct object]
}

\ex{%
Otherwise, they may be explicitly terminated to form a full sentence:
}{
[phrase-e$_a$]
}

\ex{%
More complex constructions (subclauses, sentential arguments, abstractions,
relative clauses) can be formed with the word \textbf{vy} as follows:
}{
\textbf{vy} [phrase-e$_a$]
}

\ex{%
Due to the small lexicon, many ``words'' are compounds of smaller words:
}{
[word1][word2]\ldots[wordn]
}

\ex{%
To be more specific with how words are grouped, use \textbf{u$_a$}:
}{
[word1][word2-u$_a$][word3][word4-u$_a$][word5-u$_a$u$_a$]

$\to$ [word1 word2][word3 [word4 word5]]
}

\noindent Essentially, \textbf{u$_a$} "binds" the two previous words into a new
word.

\section{Lexicon}

\subsection{Grammatical words}

\begin{tabularx}{\textwidth}{lXX}
    word/infix & description & notes \\[0.5em]

    \textbf{a$_a$} & noun/verb separator & \\
    \textbf{e$_a$} & phrase/subclause terminator & \\
    \textbf{o$_a$} & flip & \\
    \textbf{u$_a$} & compound binder & \\[0.5em]

    \textbf{a$_g$} & polar-opposite negation & \\
    \textbf{e$_g$} & nonscalar negation & \\
    \textbf{o$_g$} & augmentative & \\
    \textbf{i$_g$} & diminutive & \\[0.5em]

    \textbf{vy} & subclause introducer & can be realized as [β] \\
    \textbf{xy} & subclause "it" & from mathematical "x" \\
    \textbf{fy} & postfix flip & can be realized as [ɸ] \\
    \textbf{sy} & postfix polar-opposite negation & can be realized as [θ] \\
    \textbf{zy} & postfix nonscalar negation & can be realized as [ð] \\
\end{tabularx}

\subsection{Numbers}

TODO

\subsection{Vocabulary}

\dfn{pi}{S is me}{I, me}{from English ``me''}
\dfn{tu}{S is you}{you}{from Spanish ``tú''}
\dfn{ve}{S is this}{this}{from Kurdish ``ev'' this}
\dfn{cu}{S is that}{that}{from Turkish/Crimean Tatar ``şu'' that}
\dfn{thi}{S is that there}{that there}{from Thai ``{\thaifont ที่นั่น}'' (``tîi nân'') there}

\dfn{bve}{S is an animal}{animal}{from English ``BVetMed'' Bachelor of Veterinary Medicine}
\dfn{bzu}{S is a moon}{moon}{from French ``bzou'' werewolf}
\dfn{dva}{S is air}{air}{from Latvian ``dvaša'' breath, air}
\dfn{dzwon}{S is an ear}{ear}{from Polish ``dzwon'' bell}
\dfn{ghan}{S is music}{music, song}{from Arabic ``{\arabicfont غنى}'' (``ɣannaː'') sing}
\dfn{ghwo}{S is a tongue}{tongue}{from Greek ``γλώσσα'' tongue}
\dfn{gva}{S is a sound}{sound, noise}{from Esperanto ``gvati'' spy}
\dfn{kce}{S is water}{water, wet}{from Albanian ``kshetë'' mermaid}
\dfn{kway}{S is a star}{star, sun}{from Tupinambá ``kûarasy'' / Guarani ``kuarahy'' sun}
\dfn{kya}{S is an arm}{arm}{from Kannada ``{\kannadafont ಕೈ}'' (``kai'') hand}
\dfn{kyo}{S is a leg}{leg}{from Welsh ``coes'' leg}
\dfn{pci}{S is a dog}{dog, canine}{from Polish ``psi'' canine}
\dfn{pfe}{S is a horse}{horse}{from German ``pferd'' horse}
\dfn{pxay}{S is flesh}{flesh, meat, fruit}{from Unami ``pxàshikàn'' dried meat/jerky}
\dfn{pya}{S is a rock}{rock, stone}{from Romanian ``piatră'' stone}
\dfn{sye}{S is the back (of something)}{back}{from Italian ``schiena'' back}
\dfn{tswe}{S is a tree}{tree}{from Japanese ``棒''/``つえ'' (``tsue'') stick}
\dfn{twa}{S is a person}{person, human, he, she}{from Tagalog ``tao'' person / Vietnamese ``tao'' I, me}
\dfn{txu}{S is a digit}{digit, finger, toe}{from Aleut ``atx̂ux̂'' finger}
\dfn{zin}{S is an eye}{eye}{from Ukranian ``зіни́ця'' pupil}

\dfn{bay}{S is good}{good}{from Malay ``baik'' good}
\dfn{co}{S is hot}{hot, warm}{from French ``chaud'' warm}
\dfn{dun}{S is true}{true, correct, accurate}{from Vietnamese ``đúng'' correct}
\dfn{gzi}{S is small}{small, little}{from Unix command ``gzip''}
\dfn{pxo}{S is female}{female, girl, woman}{from Allentiac ``pxota'' girl}
\dfn{swa}{S is light}{light}{from Afrikaans ``swaar'' heavy}
\dfn{von}{S is new}{new, young}{from Serbo-Croatian ``нов'' new}
\dfn{xwa}{S is easy}{easy, simple}{from Swedish ``självklart'' obviously}
\dfn{xwi}{S is white}{white}{from Icelandic ``hvítur'' white}

\dfn{ghe}{S sleeps}{sleep}{from Proto Indo-European ``*ǵʰers-'' stiff}
\dfn{gwey}{S lives}{live, alive, survive}{from Proto Indo-European ``*gʷeiH$_{\mathit{3}}$w-'' live}

\dfn{bo}{S is sensed by O}{sense, observe, see, feel, hear, smell}{from Korean ``보다'' (``boda'') see}
\dfn{bjun}{S is enjoyed by O}{enjoy, fun}{from Czech ``bžunda'' fun}
\dfn{cye}{S is written by O}{write}{from Mandarin ``寫'' (``xiě'') write}
\dfn{djan}{S is known by O}{know}{from Hindi ``{\devanagarifont जानना }'' (``jānnā'') know}
\dfn{dway}{S is feared by O}{fear, scare}{from Cherokee ``{\cherokeefont ᎤᎾᏰᎯᏍᏗ}'' (``unayehisdi'') fear}
\dfn{fay}{S is made by O}{make, construct, create, form}{from Norman ``faithe'' do, make}
\dfn{khon}{S is consumed by O}{consume, eat, drink}{from English ``consume''}
\dfn{kfun}{S is struck by O}{strike, hit, kick, bite}{from Noone/Noni ``kfune'' strike}
\dfn{kuy}{S is given by O}{give, donate}{from Quechua ``quy'' give}
\dfn{kxay}{S is amusing to O}{amuse, funny}{from ǃXóõ ``kxʻái'' laugh}
\dfn{gan}{S is gone to by O}{go}{from Sanskrit ``{\devanagarifont गम्}'' (``gam'') go}
\dfn{pey}{S is referred to by name O}{name, call}{from Tamil ``{\tamilfont பெயர்}'' (``peyar'') name}
\dfn{tci}{S is done with instrument O}{instrument, tool, utensil}{from Lojban ``tutci -tci-'' tool}
\dfn{tfi}{S is searched for by O}{search, look for}{from Maltese ``tfittxija'' search}
\dfn{tha}{S is communicated by O}{communicate, express}{from Khmer ``{\khmerfont ថា}''  (``tʰaa'') say}
\dfn{thwe}{S is launched by O}{launch, throw, spit, eject}{from Burmese ``{\myanmarfont ထွေး }'' (``htwe:'') spit}
\dfn{xo}{S is wanted by O}{want, desire}{from Russian ``хоте́ть'' want}

\subsection{Common compounds}

\dfn{bozin}{S is seen by O}{see}{bo + zin}
\dfn{bodzwon}{S is heard by O}{hear}{bo + dzwon}
\dfn{fgk}{lol}{lol}{abbreviation of \textbf{faygvakxay}}
\dfn{gandva}{S is flown to by O}{fly}{gan + dva}
\dfn{gankce}{S is swum to by O}{swim}{gan + kce}
\dfn{ggg}{rip}{rip}{abbreviation of \textbf{ghegvagzi}}
\dfn{tcewitha}{S is a language}{language}{py + tci + tha; alt: tcifytha}
\dfn{xoghe}{S is tired}{tired}{xo + ghe}

\section{Sample texts}

TODO

\end{document}
